\documentclass{article}

\usepackage[detect-all]{siunitx}
\usepackage[littleEndian,numberFieldsOnce,numberBitsAbove]{bitpattern}
\usepackage{sidecap}
\usepackage{xcolor}
\usepackage{color}
\usepackage{listings}
\usepackage{hyperref}
\usepackage{tabu}
\usepackage{microtype}
\hypersetup{
    colorlinks,
    linkcolor={red!50!black},
    citecolor={blue!50!black},
    urlcolor={blue!80!black}
}

\sisetup{range-phrase = \text{--}}

\bpStartAtBit{9}

\definecolor{blue}{RGB}{42,0.0,255}
\definecolor{green}{RGB}{63,127,95}
\definecolor{red}{RGB}{255,0,0}

\lstdefinelanguage{HTAR9}
{
    morekeywords={
        mv,
        str,
        ld,
        fin,
        ret,
        add,
        sub,
        and,
        lshft,
        rshft,
        bcs,
        ba
    },
    sensitive=true,
    morecomment=[l]{//},
    morestring=[b]"
}

\lstset{
    language={HTAR9},
    basicstyle=\small\ttfamily,
    captionpos=b,
    extendedchars=true,
    tabsize=4,
    columns=fixed,
    keepspaces=true,
    showstringspaces=false,
    breaklines=true,
    frame=trbl,
    numbers=left,
    numberstyle=\tiny\ttfamily,
    commentstyle=\color{green},
    keywordstyle=\color{blue},
    stringstyle=\color{red},
}

\title{Specification for New Instruction Set Architecture}
\date{\today}
\author{Matthew Hatch, Desmond Toor}

\begin{document}
    \pagenumbering{roman}
    \maketitle
    \tableofcontents

    \newpage
    \pagenumbering{arabic}

    \section{Introduction}

    \paragraph{Designation}
    This specification describes the HTAR9 (Hatch-Toor Accumulator RISC 9-bit)
    architecture.

    \paragraph{Philosophy}
    The guiding philosophy of this architecture is that the machine code should
    be simple to write, simple to understand, and thus fully utilizes the
    limited 9 bits of the instructions.

    \paragraph{Realization}
    Based on this, we designed an ISA where each instruction is concise,
    requiring only one operand. This was accomplished by making the architecture
    in accumulator style. Although this limits the flexibility of each
    instruction, increasing program length, it also makes the assembly code much
    simpler to read and understand, reducing the chance of incorrect code.

    \paragraph{}
    Additionally, all memory access is done solely through the load and store
    instructions, making it trivial to implement an indirect memory access scheme
    and allowing access to the full 256 bytes of main memory.

    \section{Technical Specifications}

    \subsection{Instruction Formats}

    Because of the simplicity of the architecture, all instructions have either
    no operands or a single operand. Four instruction groups exist: \\
    {\tabulinesep=1.2mm
    \begin{tabu}{ | l | p{3cm} | l | l | }
        \hline
        Instruction Type & Operand & Format & Example \\ \hline
        simple & none & {\centering \bitpattern{opcode}[9]{}[0]/ \par} & fin \\ \hline
        register-only & register $rs$ & {\centering \bitpattern{opcode}[6]{rs}[3]{}[0]/} & mv r3 \\ \hline
        register or immediate & $rs$ or unsigned immediate & {\centering \bitpattern{op}[3]{imm/rs}[6]{}[0]/} & add r2 \\ \hline
        branch & label or signed immediate & {\centering \bitpattern{op}[3]{imm (sgnd)}[6]{}[0]/} & bch 5 \\
        \hline
    \end{tabu}}

    \subsubsection{Operand Representations}

    All operands are represented in little-endian ordering.

    \paragraph{Register-only}
    These operands are represented as a 3-bit identifier, where the identifier
    is the register's number's complement with respect to 7, e.g. r1's
    identifier is 6, that is, $110_2$.

    \paragraph{Register or Unsigned Immediate}
    These operands are represented as a 6-bit unsigned integer. Values
    \numrange{0}{55} are valid immediates. Values \numrange{56}{63} are used to
    signify registers, and thus are reserved and disallowed as
    immediates. The low 3 bits of these reserved values correspond to the
    identifiers in the above section.

    \paragraph{Signed Immediate}
    These operands are represented as a 6-bit two's complement integer. Values
    \numrange{-32}{31} are thus representable.

    \subsection{Operations}

    {\tabulinesep=1.2mm
    \begin{tabu}{ | l | l | p{4cm} | p{2.45cm} | }
        \hline
        insn & Format & Description & Condition Register \\ \hline
        mv & {\centering \bitpattern{000}[3]{000}[3]{rs}[3]{}[0]/} & moves the contents of $ra$ into $rs$; sets $ra$ to 0 & no effect \\ \hline
        str & {\centering \bitpattern{000}[3]{010}[3]{rs}[3]{}[0]/} & stores the value of $ra$ in the memory address pointed to by $rs$ &
        no effect \\ \hline
        ld & {\centering \bitpattern{000}[3]{011}[3]{rs}[3]{}[0]/} & loads the value in the memory address pointed to by $rs$ into $ra$ &
        no effect \\ \hline
        dist & {\centering \bitpattern{000}[3]{100}[3]{rs}[3]{}[0]/} & (signed) computes the absolute difference between $rs$ and $ra$ & no effect \\ \hline
        min & {\centering \bitpattern{000}[3]{101}[3]{rs}[3]{}[0]/} & computes the minimum of $rs$ and $ra$ & no effect \\ \hline
        fin & {\centering \bitpattern{000}[3]{111}[3]{000}[3]{}[0]/} & sets the done pin high to signal program completion & no effect \\ \hline
        reset & {\centering \bitpattern{000}[3]{111}[3]{001}[3]{}[0]/} & resets the program counter to 0 & no effect \\ \hline
        add & {\centering \bitpattern{001}[3]{imm/rs}[6]{}[0]/} & adds the contents of $rs$ or $imm$ to $ra$ & NOT carry-out \\ \hline
        sub & {\centering \bitpattern{010}[3]{imm/rs}[6]{}[0]/} & subtracts the contents of $rs$ or $imm$ from $ra$ & $result \neq 0$ \\ \hline
        and & {\centering \bitpattern{011}[3]{imm/rs}[6]{}[0]/} & evaluates bitwise AND between $rs$ or $imm$ and $ra$ & $result = 0$ \\ \hline
        lshft & {\centering \bitpattern{100}[3]{imm/rs}[6]{}[0]/} & shifts $ra$ left by $imm$ number of bits & $result \neq 0$ \\ \hline
        rshft & {\centering \bitpattern{101}[3]{imm/rs}[6]{}[0]/} & shifts $ra$ right by $imm$ number of bits & $result \neq 0$ \\ \hline
        bcs & {\centering \bitpattern{110}[3]{imm (sgnd)}[6]{}[0]/} & branch to label or by offset if condition register is set & no effect \\ \hline
        ba & {\centering \bitpattern{111}[3]{imm (sgnd)}[6]{}[0]/} & branch to label or by offset always & no effect \\
        \hline
    \end{tabu}}

    \subsubsection{Memory Addressing}

    All non-memory instructions use direct register addressing. The load and
    store instructions use indirect memory addressing, where the register
    operand contains the address of the memory to access. E.g. ld $r2$ treats
    the value in $r2$ as an 8-bit pointer, and loads the value in memory at the
    specified address into $ra$.

    \subsection{Internal Operands}

    \paragraph{Main Registers}
    The architecture supports 8 unsigned 8-bit registers, labelled
    $r\numrange{0}{7}$. $r0$ is aliased as $ra$, and is designated as the
    accumulator register. As noted in each opcode's description, this
    register is the only register in which arithmetic and logical operations
    are performed.

    \paragraph{Condition Register}
    An additional 1-bit condition register is set internally by all arithmetic
    and logical instructions as noted in their descriptions. The state of this
    register is used to determine whether branch instructions are followed or
    not.

    \subsection{Control Flow}

    \paragraph{Branches}
    The architecture has two instructions for implementing branches, the
    instructions $bch$ and $ba$. $bch$ is a conditional branch contingent on the
    state of the internal condition register; if it is set, the branch is taken,
    otherwise the instruction is ignored. $ba$ is an unconditional branch. Both
    branches take an offset or a label as the operand. The offset to labels is
    computed by the assembler; in hardware all branches are implemented as
    relative jumps. The maximum jump distance is 32 (backwards) or 31
    (forwards).

    \section{Examples}

    \subsection{Basic Example}

    An example of an instruction in this architecture and its machine
    translation is:
    \vspace{2.5mm}
    \newline
    sub r6 : 010111001

    \subsection{Advanced Examples}

    \subsubsection{Multiplication Problem}

    \lstinputlisting{../assembly/mult.s}

    \subsubsection{String Match Problem}

    \lstinputlisting{../assembly/string.s}

    \subsubsection{Closest Pair Problem}

    \lstinputlisting{../assembly/pair.s}

\end{document}
